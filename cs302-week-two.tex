\documentclass[14pt,aspectratio=169]{beamer}

\usepackage{pgfpages}
\usepackage{fancyvrb}
\usepackage{tikz}
\usepackage{pgfplots}

\usepackage{minted}
\usemintedstyle{tango}

\usetheme{auriga}
\usecolortheme{auriga}

\setbeamercolor{background canvas}{bg=lightgray}

% define some colors for a consistent theme across slides
\definecolor{red}{RGB}{181, 23, 0}
\definecolor{blue}{RGB}{0, 118, 186}
\definecolor{gray}{RGB}{146, 146, 146}

\title{Web Development: \\ How the Web Works}

\author{{\bf Gregory M. Kapfhammer}}

\institute[shortinst]{{\bf Department of Computer Science, Allegheny College}}

\begin{document}

{
  \setbeamercolor{page number in head/foot}{fg=background canvas.bg}
  \begin{frame}
    \titlepage
  \end{frame}
}

%% Slide
%
\begin{frame}{Technical Question}
  %
  \hspace*{.25in}
  %
  \vspace*{.2in}
  %
  \begin{center}
    %
    {\large How do I connect the documented behavior of a web browser and a
      web client (e.g., routing, request, and response) to their observed
      behavior when using a simplified hypertext-transfer protocol (HTTP) server
    and a standard web browser?}
    %
  \end{center}
  %
  \vspace{1ex}
  %
  \begin{center}
    %
    \small Let's learn more about complexities associated with the development,
    deployment, and maintenance of modern web sites!
    %
  \end{center}
  %
\end{frame}

% Slide
%
\begin{frame}{The Web is a Complicated Ecosystem}
%
  \begin{itemize}
    %
    \item Great web sites seem so simple, but they are complex!
      %
    \item Using many languages, tools, and techniques
      \begin{itemize}
        \item Markdown, HTML, CSS, JavaScript
        \item Design, implementation, and testing
        \item Deployment, maintenance, monitoring
        \item Graphic design and multimedia
      \end{itemize}
      %
    \item Lines often blur between web, desktop, mobile!
      %
  \end{itemize}
%
\end{frame}

% Slide
%
\begin{frame}{Comparing and Contrasting Configurations}
%
  \begin{itemize}
    %
    \item Numerous terms sound the same but are very different!
      %
    \item Compare and contrast these configurations
      \begin{itemize}
        \item Intranet versus internet
        \item Static versus dynamic web site
        \item Dynamic server-side versus client-side
      \end{itemize}
      %
    \item What type of server is ``{\tt python -m http.server
      src/html/learning-objectives}''?
      %
  \end{itemize}
%
\end{frame}

% Slide
%
\begin{frame}{Client-Server Web Architecture}
  %
  \begin{itemize}
    %
    \item Clients and servers communicate by ``request-response''
      %
      \vspace*{-.2in}
      %
    \item Repeated communication between a client and a server
      \begin{itemize}
        \item {\bf Client}: desktop, laptop, tablet, phone
        \item {\bf Client Request}: pages or web elements
        \item {\bf Server}: repository and command center
        \item {\bf Server Response}: pages or web elements
      \end{itemize}
      %
      \vspace*{-.2in}
      %
    \item ``Keep the computation close to the data'' is a good rule of thumb for
      ensuring that a web application minimizes response time and reduces
      network traffic
      %
  \end{itemize}
  %
\end{frame}

% Slide
%
\begin{frame}{Internet Routing for Web Sites}
  %
  \begin{itemize}
    %
    \item Data between client and server discretized into packets
      %
    \item Routing steps between the client and the server
      %
      \begin{itemize}
        %
        \item Your laptop to your local ISP
          %
        \item The local ISP to regional networks
          %
        \item Regional networks to exchange points
          %
        \item The server to which you intended to connect
          %
      \end{itemize}
      %
      \vspace*{-.2in}
      %
    \item How do you resolve the human-readable name of the server to its actual
      address? You need to use the domain name system (DNS) to perform name
      resolution!
      %
  \end{itemize}
  %
\end{frame}

% Slide
%
\begin{frame}[fragile]
  \frametitle{Network Diagnostics for a Client}
  \normalsize
  \hspace*{-.65in}
  \begin{minipage}{6in}
    \vspace*{.25in}
    \begin{minted}[mathescape, numbersep=5pt, fontsize=\footnotesize]{console}
        inet 192.168.0.180  netmask 255.255.255.0  broadcast 192.168.0.255
        inet6 fe80::dff1:eed8:49fb:1265  prefixlen 64  scopeid 0x20<link>
        ether cc:f9:e4:9e:b0:97  txqueuelen 1000  (Ethernet)
        RX packets 11865  bytes 7807225 (7.4 MiB)
        RX errors 0  dropped 0  overruns 0  frame 0
        TX packets 9960  bytes 1614279 (1.5 MiB)
        TX errors 0  dropped 0 overruns 0  carrier 0  collisions 0
    \end{minted}
  \end{minipage}
  \vspace*{.25in}
  \begin{center}
    %
    \normalsize \noindent What is the IP address of this computer? \\
    \normalsize \noindent What is the MAC address of this computer? \\
    \normalsize \noindent Does this computer transmit packets reliably? \\
    %
  \end{center}
  %
\end{frame}

\end{document}
