\documentclass[14pt,aspectratio=169]{beamer}

\usepackage{pgfpages}
\usepackage{fancyvrb}

\usepackage{tikz}
\usepackage{pgfplots}

\usepackage{minted}
\usemintedstyle{tango}

\usepackage{graphicx}

\usetheme{auriga}
\usecolortheme{auriga}

\setbeamercolor{background canvas}{bg=lightgray}

% define some colors for a consistent theme across slides
\definecolor{red}{RGB}{181, 23, 0}
\definecolor{blue}{RGB}{0, 118, 186}
\definecolor{gray}{RGB}{146, 146, 146}

\title{Web Development: \\ Using Color and Web Media}

\author{{\bf Gregory M. Kapfhammer}}

\institute[shortinst]{{\bf Department of Computer Science, Allegheny College}}

\begin{document}

{
  \setbeamercolor{page number in head/foot}{fg=background canvas.bg}
  \begin{frame}
    \titlepage
  \end{frame}
}

% Slide
%
\begin{frame}{Technical Question}
  %
  \hspace*{.25in}
  %
  \vspace*{.2in}
  %
  \begin{minipage}{5in}
    %
    \begin{center}
      %
      {\large How can I use HTML and CSS source code to design and implement
      multicolumn and responsive web page layouts?}
      %
    \end{center}
    %
  \end{minipage}
  %
  \vspace{2ex}
  %
  \begin{center}
    %
    \small Let's learn how to combine CSS and HTML to mobile-ready layouts for
    our web pages! We will also explore the benefits of using
    responsive web design frameworks. Make sure to review all previous content!\\
    %
  \end{center}
  %
\end{frame}

% Slide
%
\begin{frame}{The Normal Flow of Web Page Layout}
  %
  \begin{itemize}
    %
    \item How does the web browser normally layout the block-level and inline
      elements from left to right and top to bottom?
      %
      \vspace*{-.15in}
      %
    \item The fundamental components of a web page:
      %
      \begin{itemize}
        %
        \item {\bf Block-level elements}: content contained on their own line
          %
        \item {\bf Inline elements}: content displayed within an existing line
          %
          \begin{itemize}
            %
            \item {\bf Replaced}: content defined by an external resource
              %
            \item {\bf Non-replaced}: content defined by an in-document
              resources
              %
          \end{itemize}
          %
      \end{itemize}
      %
      \vspace*{-.2in}
      %
    \item How can you tell if a component is block-level or not?
      %
      \vspace*{-.2in}
      %
    \item How can you tell if a component is inline or not?
      %
      \vspace*{-.2in}
      %
    \item When is an element replaced versus non-replaced?
      %
  \end{itemize}
  %
\end{frame}

% Slide
%
\begin{frame}[fragile]
  \frametitle{Identifying Block-Level and Inline Content}
  \normalsize
  \begin{minipage}{6in}
    \vspace*{.1in}
    \begin{minted}[mathescape, numbersep=5pt, fontsize=\large]{html}
<h3>Photograph Reviews</h3>
<blockquote>
  <p><b>By Ricardo on
     <time>February 8, 2018</time></b>
  </p>
  <p>That is a great photograph!</p>
  <p>I would describe this as
     <i class="em em---1"></i></p>
</blockquote>
    \end{minted}
  \end{minipage}
%
\end{frame}

% Slide
%
\begin{frame}[fragile]
  \frametitle{Finding Replaced and Non-Replaced Content}
  \normalsize
  \begin{minipage}{6in}
    \vspace*{.2in}
    \begin{minted}[mathescape, numbersep=5pt, fontsize=\large]{html}

  <a title="A Delta 757 lands at LAX
      on January 29th. #latergram"
  href="https://flickr.com/photos/
        pt737swa/28371915769">
      <img src="img/plane.jpg"/>
  </a>

    \end{minted}
  \end{minipage}
  %
  \vspace*{.1in}
  %
  \begin{center}
    Which content would we categorize as being replaced?
  \end{center}
\end{frame}

% Slide
%
\begin{frame}{Models for Representing Colors on the Web}
  %
  \begin{itemize}
    %
    \item We encode a color as a sequence of numbers
      %
      \vspace*{-.2in}
      %
    \item What are the benefits of encoding color as a number?
      %
      \vspace*{-.2in}
      %
    \item Popular color models for web pages:
      %
      \begin{itemize}
        %
        \item {\bf RGB}: Use the hexadecimal {\tt \#RRGGBB} format
          %
        \item {\bf CYMK}: Use cyan, magenta, yellow, and black for print
         %
        \item {\bf HSL}: Use hue, saturation, and lightness in CSS3
         %
        \item We adopt the RGB color model for web development!
         %
      \end{itemize}
      %
      \vspace*{-.25in}
      %
    \item You can also manipulate a color by changing its opacity
      %
      \vspace*{-.25in}
      %
    \item You can display colors differently through use of a gradient
      %
  \end{itemize}
  %
\end{frame}

\end{document}
