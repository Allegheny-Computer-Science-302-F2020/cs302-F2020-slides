\documentclass[14pt,aspectratio=169]{beamer}

\usepackage{pgfpages}
\usepackage{fancyvrb}

\usepackage{tikz}
\usepackage{pgfplots}

\usepackage{minted}
\usemintedstyle{tango}

\usepackage{graphicx}

\usetheme{auriga}
\usecolortheme{auriga}

\setbeamercolor{background canvas}{bg=lightgray}

% define some colors for a consistent theme across slides
\definecolor{red}{RGB}{181, 23, 0}
\definecolor{blue}{RGB}{0, 118, 186}
\definecolor{gray}{RGB}{146, 146, 146}

\title{Web Development: \\ Creating Tables and Forms}

\author{{\bf Gregory M. Kapfhammer}}

\institute[shortinst]{{\bf Department of Computer Science, Allegheny College}}

\begin{document}

{
  \setbeamercolor{page number in head/foot}{fg=background canvas.bg}
  \begin{frame}
    \titlepage
  \end{frame}
}

% Slide
%
\begin{frame}{Technical Question}
  %
  \hspace*{.25in}
  %
  \vspace*{.1in}
  %
  \begin{minipage}{4.5in}
  %
  \begin{center}
    %
    {\large How can I follow industry best practices to use both HTML and CSS to
    implement tables that display formatted content and forms that receive and
  verify user input?}
    %
  \end{center}
  %
  \end{minipage}
  %
  \vspace{2ex}
  %
  \begin{center}
    %
    \small Let's learn how to combine CSS and HTML to create tables and forms!
    %
  \end{center}
  %
\end{frame}

% Slide
%
\begin{frame}{Table Structure in HTML}
  %
  \begin{itemize}
    %
    \item An HTML table consists of rows and columns
      %
      \vspace*{-.1in}
      %
    \item HTML tags used in an HTML table
      %
      \begin{itemize}
        %
        \item {\tt <table>}: Define the boundaries of the table
          %
        \item {\tt <tr>}: Define a row of the table
          %
        \item {\tt <td>}: Define a table data cell in a row
          %
      \end{itemize}
      %
      \vspace*{-.2in}
      %
    \item What type of content is best represented in a table?
      %
      \vspace*{-.2in}
      %
    \item What are not the most appropriate uses of a table?
      %
      \vspace*{-.2in}
      %
    \item What are the challenges associated with creating a table?
      %
  \end{itemize}
  %
\end{frame}

% Slide
%
\begin{frame}[fragile]
  \frametitle{Programming Tables Using HTML}
  \normalsize
  \hspace*{.25in}
  \begin{minipage}{6in}
    \vspace*{.2in}
    \begin{minted}[mathescape, numbersep=5pt, fontsize=\large]{css}
<table>
  <tr>
    <td>Passport</td>
    <td>Magazine</td>
    <td>Pants</td>
  </tr>
</table>
    \end{minted}
  \end{minipage}
  \vspace*{.05in}
  \begin{center}
    %
    \noindent What are the HTML tags used in this table?\\
    \noindent How would this table look when rendered by the browser?\\
    %
  \end{center}
  %
\end{frame}

% Slide
%
\begin{frame}[fragile]
  \frametitle{Styling Tables Using CSS}
  \normalsize
  \hspace*{.25in}
  \begin{minipage}{6in}
    \vspace*{.1in}
    \begin{minted}[mathescape, numbersep=5pt, fontsize=\large]{css}
table {
  color: #333;
  width: 600px;
  border: solid #777;
  border-width: 1pt 0pt 0pt 1pt;
  border-spacing: 0;
  margin: 30px;
  padding: 5px 5px 5px 5px;
}
    \end{minted}
  \end{minipage}
  %
\end{frame}

% Slide
%
\begin{frame}[fragile]
  \frametitle{Styling HTML Tables Using CSS}
  \normalsize
  \hspace*{.25in}
  \begin{minipage}{6in}
    \vspace*{.2in}
    \begin{minted}[mathescape, numbersep=5pt, fontsize=\large]{css}
td, th {
  border: 1px solid transparent;
  height: 30px;
}
    \end{minted}
  \end{minipage}
  \vspace*{.05in}
  \begin{center}
    %
    \noindent What are the HTML tags that are styled by this CSS?\\
    %
    \noindent How does this CSS change the style of the HTML table?\\
    %
    \noindent What is the purpose of the {\tt border} styling code?\\
    %
  \end{center}
  %
\end{frame}

% Slide
%
\begin{frame}[fragile]
  \frametitle{Styling Table Headers and Cells with CSS}
  \normalsize
  \hspace*{.25in}
  \begin{minipage}{6in}
    \vspace*{.2in}
    \begin{minted}[mathescape, numbersep=5pt, fontsize=\large]{css}
th {
  font-weight: bold;
}
td {
  text-align: center;
}
    \end{minted}
  \end{minipage}
  \vspace*{.05in}
  \begin{center}
    %
    \noindent What is the difference between {\tt <td>} and {\tt <th>}?\\
    %
    \noindent How does this CSS change the style of the HTML table?\\
    %
  \end{center}
  %
\end{frame}

% Slide
%
\begin{frame}[fragile]
  \frametitle{Adding Patterns to Table Rows}
  \normalsize
  \hspace*{.25in}
  \begin{minipage}{6in}
    \vspace*{.2in}
    \begin{minted}[mathescape, numbersep=5pt, fontsize=\large]{css}
tr:nth-child(even) td {
            background: #F1F1F1;
            }
tr:nth-child(odd) td {
            background: #FEFEFE;
            }
    \end{minted}
  \end{minipage}
  \vspace*{.05in}
  \begin{center}
    %
    \noindent What is the purpose of the {\tt nth-child} designator?\\
    %
    \noindent How does this CSS change the style of the HTML table?\\
    %
  \end{center}
  %
\end{frame}

% Slide
%
\begin{frame}[fragile]
  \frametitle{Adding Interactivity to Table Rows}
  \normalsize
  \hspace*{.25in}
  \begin{minipage}{6in}
    \vspace*{.2in}
    \begin{minted}[mathescape, numbersep=5pt, fontsize=\large]{css}
tr td:hover {
            background: #777;
            color: #FFF;
            }
    \end{minted}
  \end{minipage}
  \vspace*{.05in}
  \begin{center}
    %
    \noindent What is the difference between {\tt <tr>} and {\tt <td>}?\\
    %
    \noindent When does a person ``hover'' on a table in a web page?\\
    %
    \noindent How does this CSS add a small measure of table interaction?\\
    %
  \end{center}
  %
\end{frame}

% Slide
%
\begin{frame}{Using Tables for the Layout of Content}
  %
  \begin{itemize}
    %
    \item Common approach before the existence of Flexbox and CSS Grid, now less
      widely adopted on modern sites
      %
      \vspace*{-.1in}
      %
    \item Problems associated with using tables for content layout
      %
      \begin{itemize}
        %
        \item Dramatically increases the size of the HTML document
          %
        \item Leverages tags that do not convey meaning about the content
          %
        \item Yields content that is not accessible to screen readers
          %
      \end{itemize}
      %
      \vspace*{-.2in}
      %
    \item What is a better alternative to using tables for layout?
      %
      \vspace*{-.2in}
      %
    \item What are the differences between CSS Grid and Flexbox?
      %
      \vspace*{-.2in}
      %
    \item What are the differences between Bootstrap and Foundation? Bootstrap
      and CSS Grid?
      %
  \end{itemize}
  %
\end{frame}

\end{document}
