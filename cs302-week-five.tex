\documentclass[14pt,aspectratio=169]{beamer}

\usepackage{pgfpages}
\usepackage{fancyvrb}

\usepackage{tikz}
\usepackage{pgfplots}

\usepackage{minted}
\usemintedstyle{tango}

\usepackage{graphicx}

\usetheme{auriga}
\usecolortheme{auriga}

\setbeamercolor{background canvas}{bg=lightgray}

% define some colors for a consistent theme across slides
\definecolor{red}{RGB}{181, 23, 0}
\definecolor{blue}{RGB}{0, 118, 186}
\definecolor{gray}{RGB}{146, 146, 146}

\title{Web Development: \\ Programming in CSS}

\author{{\bf Gregory M. Kapfhammer}}

\institute[shortinst]{{\bf Department of Computer Science, Allegheny College}}

\begin{document}

{
  \setbeamercolor{page number in head/foot}{fg=background canvas.bg}
  \begin{frame}
    \titlepage
  \end{frame}
}

%% Slide
%
\begin{frame}{Technical Question}
  %
  \hspace*{.25in}
  %
  \vspace*{.1in}
  %
  \begin{center}
    %
    {\large How can I follow industry best practices for implementing syntactically valid
      cascading style sheets (CSS) files that correctly use selectors, properties,
    and media queries to modify the display of text-based and image content?}
    %
  \end{center}
  %
  \vspace{1ex}
  %
  \begin{center}
    %
    \small Let's learn more about how to use CSS to write useful style sheets!
    %
  \end{center}
  %
\end{frame}

\end{document}
