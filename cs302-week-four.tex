\documentclass[14pt,aspectratio=169]{beamer}

\usepackage{pgfpages}
\usepackage{fancyvrb}
\usepackage{tikz}
\usepackage{pgfplots}

\usepackage{minted}
\usemintedstyle{tango}

\usetheme{auriga}
\usecolortheme{auriga}

\setbeamercolor{background canvas}{bg=lightgray}

% define some colors for a consistent theme across slides
\definecolor{red}{RGB}{181, 23, 0}
\definecolor{blue}{RGB}{0, 118, 186}
\definecolor{gray}{RGB}{146, 146, 146}

\title{Web Development: \\ Programming in CSS}

\author{{\bf Gregory M. Kapfhammer}}

\institute[shortinst]{{\bf Department of Computer Science, Allegheny College}}

\begin{document}

{
  \setbeamercolor{page number in head/foot}{fg=background canvas.bg}
  \begin{frame}
    \titlepage
  \end{frame}
}

%% Slide
%
\begin{frame}{Technical Question}
  %
  \hspace*{.25in}
  %
  \vspace*{.1in}
  %
  \begin{center}
    %
    {\large How can I follow industry best practices for implementing syntactically valid
      cascading style sheets (CSS) files that correctly use selectors, properties,
    and media queries to modify the display of text-based and image content?}
    %
  \end{center}
  %
  \vspace{1ex}
  %
  \begin{center}
    %
    \small Let's learn more about how to use CSS to write useful style sheets!
    %
  \end{center}
  %
\end{frame}

% Slide
%
\begin{frame}{Web Pages with HTML and CSS Code}
%
  \begin{itemize}
    %
    \item CSS stands for the language for cascading style sheets
      %
    \item HTML and CSS are major components of the modern web
      \begin{itemize}
        \item Markdown, HTML, CSS, JavaScript
        \item Design, implementation, and testing
        \item Deployment, maintenance, monitoring
        \item Linting and testing tools for all languages
      \end{itemize}
      %
    \item What are the style sheets described as ``cascading''?
      %
  \end{itemize}
%
\end{frame}

% Slide
%
\begin{frame}{Correctly Using Both HTML and CSS}
%
  \begin{itemize}
    %
    \item HTML should not handle formatting or presentation
      %
      \vspace*{-.2in}
      %
    \item CSS can describe the appearance of HTML elements
      %
      \vspace*{-.15in}
      %
    \item Benefits of using CSS to style elements
      \begin{itemize}
        \item Improved control over formatting
        \item Improved site maintainability
        \item Improved accessibility
        \item Improved page download speed
        \item Improved output flexibility
      \end{itemize}
      %
      \vspace*{-.2in}
      %
    \item CSS enables {\em mobile-ready} or {\em responsive} web pages
  \end{itemize}
%
\end{frame}

% Slide
%
\begin{frame}{Versions and Adoption of CSS}
%
  \begin{itemize}
    %
    \item Distinction between {\em data} and {\em meta-data} and {\em style}
      %
      \vspace*{-.1in}
      %
    \item Versions and adoption of CSS
      \begin{itemize}
        \item Should style be expressed in JavaScript?
        \item Should style be expressed by key value pairs?
        \item Should styling involve colors and fonts?
        \item How should web page styling be organized?
      \end{itemize}
      %
      \vspace*{-.2in}
      %
    \item Check sites like \url{https://caniuse.com/} to learn more about the
      current features of CSS3 and its browser support. Different browsers have
      different support!
      %
  \end{itemize}
%
\end{frame}

% Slide
%
\begin{frame}{Syntax and Semantics of CSS}
%
  \begin{itemize}
    %
    \item A {\em selector} identifies a matching HTML element
      %
      \vspace*{-.2in}
      %
    \item Key-value pairs define properties associated with values
      %
      \vspace*{-.15in}
      %
    \item Summary of the components in CSS files
      \begin{itemize}
        \item Selectors
        \item Properties
        \item Values
        \item Organized into style sheets
        \item Located in three different places
      \end{itemize}
      %
      \vspace*{-.2in}
      %
    \item CSS content can be {\em inline} or {\em embedded} or {\em external} to web pages.
      What are the trade-offs in these approaches?
  \end{itemize}
%
\end{frame}

\end{document}
