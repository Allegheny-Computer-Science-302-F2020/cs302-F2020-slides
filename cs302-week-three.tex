\documentclass[14pt,aspectratio=169]{beamer}

\usepackage{pgfpages}
\usepackage{fancyvrb}
\usepackage{tikz}
\usepackage{pgfplots}

\usepackage{minted}
\usemintedstyle{tango}

\usetheme{auriga}
\usecolortheme{auriga}

\setbeamercolor{background canvas}{bg=lightgray}

% define some colors for a consistent theme across slides
\definecolor{red}{RGB}{181, 23, 0}
\definecolor{blue}{RGB}{0, 118, 186}
\definecolor{gray}{RGB}{146, 146, 146}

\title{Web Development: \\ Programming in HTML}

\author{{\bf Gregory M. Kapfhammer}}

\institute[shortinst]{{\bf Department of Computer Science, Allegheny College}}

\begin{document}

{
  \setbeamercolor{page number in head/foot}{fg=background canvas.bg}
  \begin{frame}
    \titlepage
  \end{frame}
}

%% Slide
%
\begin{frame}{Technical Question}
  %
  \hspace*{.25in}
  %
  \vspace*{.2in}
  %
  \begin{center}
    %
    {\large How can I follow industry best practices for implementing
      syntactically valid hypertext markup language (HTML) files that correctly
    use content and tags to display text-based and image content?}
    %
  \end{center}
  %
  \vspace{1ex}
  %
  \begin{center}
    %
    \small Let's learn more about how to use HTML to write syntactically valid markup!
    %
  \end{center}
  %
\end{frame}

% Slide
%
\begin{frame}{Creating Web Sites Using HTML and ...}
%
  \begin{itemize}
    %
    \item HTML standards for the hypertext markup language
      %
    \item HTML is one of the many components in the modern web
      \begin{itemize}
        \item Markdown, HTML, CSS, JavaScript
        \item Design, implementation, and testing
        \item Deployment, maintenance, monitoring
        \item Linting and testing tools for all languages
      \end{itemize}
      %
    \item HTML source code must have correct {\em syntax} and {\em semantics}
      %
  \end{itemize}
%
\end{frame}

% Slide
%
\begin{frame}{Comparing and Contrasting Configurations}
%
  \begin{itemize}
    %
    \item Numerous terms sound the same but are very different!
      %
    \item Compare and contrast these configurations
      \begin{itemize}
        \item Intranet versus internet
        \item Static versus dynamic web site
        \item Dynamic server-side versus client-side
      \end{itemize}
      %
    \item What type of server is ``{\tt python -m http.server
      src/html/learning-objectives}''?
      %
  \end{itemize}
%
\end{frame}

% Slide
%
\begin{frame}{Client-Server Web Architecture}
  %
  \begin{itemize}
    %
    \item Clients and servers communicate by ``request-response''
      %
      \vspace*{-.2in}
      %
    \item Repeated communication between a client and a server
      \begin{itemize}
        \item {\bf Client}: desktop, laptop, tablet, phone
        \item {\bf Client Request}: pages or web elements
        \item {\bf Server}: repository and command center
        \item {\bf Server Response}: pages or web elements
        \item {\bf Protocols}: HTTP, HTTPS, SSL, TCP/IP
      \end{itemize}
      %
      \vspace*{-.2in}
      %
    \item ``Keep the computation close to the data''
      ensures that a web application minimizes response time and reduces
      network traffic. Data compression? Overall trade-offs?
      %
  \end{itemize}
  %
\end{frame}

% Slide
%
\begin{frame}{Internet Routing for Web Sites}
  %
  \begin{itemize}
    %
    \item Data between client and server discretized into packets
      %
    \item Routing steps between the client and the server
      %
      \begin{itemize}
        %
        \item Your laptop to your local ISP
          %
        \item The local ISP to regional networks
          %
        \item Regional networks to exchange points
          %
        \item The server to which you intended to connect
          %
      \end{itemize}
      %
      \vspace*{-.2in}
      %
    \item How do you resolve the human-readable name of the server to its actual
      address? You need to use the domain name system (DNS) to perform name
      resolution!
      %
  \end{itemize}
  %
\end{frame}

% Slide
%
\begin{frame}[fragile]
  \frametitle{Network Diagnostics for a Client}
  \normalsize
  \hspace*{-.65in}
  \begin{minipage}{6in}
    \vspace*{.25in}
    \begin{minted}[mathescape, numbersep=5pt, fontsize=\footnotesize]{text}
        inet 192.168.0.180  netmask 255.255.255.0  broadcast 192.168.0.255
        inet6 fe80::dff1:eed8:49fb:1265  prefixlen 64  scopeid 0x20<link>
        ether cc:f9:e4:9e:b0:97  txqueuelen 1000  (Ethernet)
        RX packets 11865  bytes 7807225 (7.4 MiB)
        RX errors 0  dropped 0  overruns 0  frame 0
        TX packets 9960  bytes 1614279 (1.5 MiB)
        TX errors 0  dropped 0 overruns 0  carrier 0  collisions 0
    \end{minted}
  \end{minipage}
  \vspace*{.25in}
  \begin{center}
    %
    \normalsize \noindent What is the IP address of this computer? \\
    \normalsize \noindent What is the MAC address of this computer? \\
    \normalsize \noindent Does this computer transmit packets reliably? \\
    %
  \end{center}
  %
\end{frame}

% Slide
%
\begin{frame}[fragile]
  \frametitle{Client Lookups for the Server Address}
  \normalsize
  \hspace*{-.2in}
  \begin{minipage}{6in}
    \vspace*{.15in}
    \begin{minted}[mathescape, numbersep=5pt, fontsize=\footnotesize]{text}
; <<>> DiG 9.16.6 <<>> github.com
;; global options: +cmd
;; Got answer:
;; ->>HEADER<<- opcode: QUERY, status: NOERROR, id: 28755
;; flags: qr rd ra; QUERY: 1, ANSWER: 1, AUTHORITY: 0, ADDITIONAL: 1

;; OPT PSEUDOSECTION:
; EDNS: version: 0, flags:; udp: 4096
;; QUESTION SECTION:
;github.com.                    IN      A

;; ANSWER SECTION:
github.com.             1       IN      A       140.82.114.3
    \end{minted}
  \end{minipage}
  \vspace*{0in}
  \begin{center}
    %
    \normalsize \noindent What is the IP address of {\tt github.com}? \\
    %
  \end{center}
  %
\end{frame}

% Slide
%
\begin{frame}[fragile]
  \frametitle{Domain Name Server for a Client}
  \normalsize
  \hspace*{.15in}
  \begin{minipage}{6in}
    \vspace*{.25in}
    \begin{minted}[mathescape, numbersep=5pt, fontsize=\footnotesize]{text}
;; Query time: 20 msec
;; SERVER: 72.23.14.75#53(72.23.14.75)
;; WHEN: Tue Sep 08 12:50:01 EDT 2020
;; MSG SIZE  rcvd: 55
    \end{minted}
  \end{minipage}
  \vspace*{.25in}
  \begin{center}
    %
    \normalsize \noindent What is the IP address of the computer's DNS server? \\
    \normalsize \noindent What role does DNS play in the overall routing process? \\
    \normalsize \noindent Can we learn more about packet transmission performance? \\
    %
  \end{center}
  %
\end{frame}

% Slide
%
\begin{frame}[fragile]
  \frametitle{Network Communication Costs in Milliseconds}
  \normalsize
  \hspace*{-.1in}
  \begin{minipage}{6in}
    \vspace*{.15in}
    \begin{minted}[mathescape, numbersep=5pt, fontsize=\scriptsize]{text}
 Host                            Loss%   Snt   Last   Avg  Best  Wrst StDev
 1. _gateway                      0.0%     5    1.1   1.8   1.1   3.5   1.0
 2. dynamic-acs-72-23-183-1.zoom  0.0%     5    8.6  10.9   8.6  16.1   3.2
 3. 10.201.22.66                  0.0%     5   19.5  13.8  10.4  19.5   4.1
 4. 10.200.2.61                   0.0%     5   10.8  12.3   9.9  14.8   2.0
 5. 10.200.3.34                   0.0%     5   12.4  13.5  11.7  16.2   1.8
 6. 10.200.2.117                  0.0%     5   11.4  14.0  11.4  15.3   1.6
 7. ce-0-6-0-1.r01.nycmny17.us.b  0.0%     5   76.9  39.0  28.7  76.9  21.2
 8. ae-2.r21.nwrknj03.us.bb.gin. 20.0%     5   29.7  30.2  29.7  30.7   0.6
 9. ae-3.r25.asbnva02.us.bb.gin. 20.0%     5   36.0  42.7  33.9  64.8  14.8
10. ae-7.r06.asbnva02.us.bb.gin.  0.0%     5   35.1  53.4  35.1 121.0  37.9
11. ce-0-6-0-1.r06.asbnva02.us.c  0.0%     5   32.2  40.5  32.2  61.2  11.8
15. lb-140-82-112-4-iad.github.c  0.0%     4   33.2  32.0  30.5  33.2   1.2

    \end{minted}
  \end{minipage}
  \vspace*{0in}
  \begin{center}
    %
    \normalsize \noindent Was there a lot of packet loss between client and
    server? \\
    %
    \normalsize \noindent Was data quickly routed between client and server? \\
    %
    \normalsize \noindent Does data transmission exhibit a lot of variability? \\
    %
  \end{center}
  %
\end{frame}

% Slide
%
\begin{frame}{Uniform Resource Locators for Web Sites}
  %
  \begin{itemize}
    %
    \item A standard way to represent a human-readable address
      %
    \item The URL contains several standard components
      %
      \begin{itemize}
        %
        \item Protocol
          %
        \item Port
          %
        \item Path
          %
        \item Query string
          %
        \item Fragments
          %
      \end{itemize}
      %
      \vspace*{-.2in}
      %
    \item Can you find these components in the URLs in web sites?
      Do all URLs always use all of these components?
      %
  \end{itemize}
  %
\end{frame}

% Slide
%
\begin{frame}{Hypertext Transfer Protocol}
  %
  \begin{itemize}
    %
    \item {\tt GET} requests normally involve requesting a resource
      %
    \vspace*{-.1in}
    \item {\tt POST} requests normally involve submitting content
      %
    \vspace*{-.1in}
    \item Examples of HTTP response codes
      %
      \begin{itemize}
        %
        \item 200: OK request for resource
          %
        \item 301 or 307: Moved permanently or temporarily
          %
        \item 401: Unauthorized access
          %
        \item 404: Resource not found
          %
        \item 500: Internal server error
          %
      \end{itemize}
      %
      \vspace*{-.2in}
      %
    \item Can you find these response codes in your server logs?
      %
  \end{itemize}
  %
\end{frame}

% Slide
%
\begin{frame}[fragile]
  \frametitle{Response Codes for HTTP Servers}
  \normalsize
  \hspace*{-.15in}
  \begin{minipage}{6in}
    \vspace*{.25in}
    \begin{minted}[mathescape, numbersep=5pt, fontsize=\scriptsize]{text}
  Serving HTTP on 0.0.0.0 port 8000 (http://0.0.0.0:8000/) ...
  127.0.0.1 - - [08/Sep/2020 12:47:17] "GET / HTTP/1.1" 200 -
  127.0.0.1 - - [08/Sep/2020 12:47:17] code 404, message File not found
  127.0.0.1 - - [08/Sep/2020 12:47:17] "GET /favicon.ico HTTP/1.1" 404 -
  127.0.0.1 - - [08/Sep/2020 12:47:19] "GET /web-development.html HTTP/1.1" 200 -
    \end{minted}
  \end{minipage}
  \vspace*{.25in}
  \begin{center}
    %
    \normalsize \noindent What is the IP address of this server? \\
    \normalsize \noindent What are some of the evident return codes? \\
    \normalsize \noindent What kind of content can an HTTP server return? \\
    %
  \end{center}
  %
\end{frame}

% Slide
%
\begin{frame}{Features of Modern Web Browsers}
  %
  \begin{itemize}
    %
    \item Web browsers operate as the clients of web servers
      %
      \vspace*{-.1in}
      %
    \item Common features of modern web browsers
      %
      \begin{itemize}
        %
        \item HTML and CSS parser
          %
        \item Rendering engine
          %
        \item File storage cache
          %
        \item Core features
          %
        \item Plugins or extensions
          %
      \end{itemize}
      %
      \vspace*{-.2in}
      %
    \item What are the trade-offs associated with these features? For instance,
      how does the cache influence bandwidth consumption and web page
      correctness?
      %
  \end{itemize}
  %
\end{frame}

% Slide
%
\begin{frame}{Features of Modern Web Servers}
  %
  \begin{itemize}
    %
    \item Accepts, processes, and responds to HTTP requests
      %
      \vspace*{-.1in}
      %
    \item Application ``stacks'' associated with web servers
      %
      \begin{itemize}
        %
        \item LAMP stack
          %
        \item WISA stack
          %
        \item MEAN stack
          %
        \item Database application
          %
        \item Scripting software
          %
      \end{itemize}
      %
      \vspace*{-.2in}
      %
    \item Examples of popular web servers: Apache, NGINX, and Caddy. What
      features do they share in common? How are they different? What are their
      trade-offs?
      %
  \end{itemize}
  %
\end{frame}

\end{document}
