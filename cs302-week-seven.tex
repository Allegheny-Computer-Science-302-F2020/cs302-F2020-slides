\documentclass[14pt,aspectratio=169]{beamer}

\usepackage{pgfpages}
\usepackage{fancyvrb}

\usepackage{tikz}
\usepackage{pgfplots}

\usepackage{minted}
\usemintedstyle{tango}

\usepackage{graphicx}

\usetheme{auriga}
\usecolortheme{auriga}

\setbeamercolor{background canvas}{bg=lightgray}

% define some colors for a consistent theme across slides
\definecolor{red}{RGB}{181, 23, 0}
\definecolor{blue}{RGB}{0, 118, 186}
\definecolor{gray}{RGB}{146, 146, 146}

\title{Web Development: \\ Using Color and Web Media}

\author{{\bf Gregory M. Kapfhammer}}

\institute[shortinst]{{\bf Department of Computer Science, Allegheny College}}

\begin{document}

{
  \setbeamercolor{page number in head/foot}{fg=background canvas.bg}
  \begin{frame}
    \titlepage
  \end{frame}
}

% Slide
%
\begin{frame}{Technical Question}
  %
  \hspace*{.25in}
  %
  \vspace*{.1in}
  %
  \begin{minipage}{5in}
    %
    \begin{center}
      %
      {\large How can I use HTML and CSS source code and media creation software to
      effectively add color and media such as images to a web site?}
      %
    \end{center}
    %
  \end{minipage}
  %
  \vspace{2ex}
  %
  \begin{center}
    %
    \small Let's learn how to combine CSS and HTML to create color and images!\\
    \small We will also explore the benefits of using external software to
    create media !\\
    \small Since web development in cumulative, please review all previous content!\\
    %
  \end{center}
  %
\end{frame}

% Slide
%
\begin{frame}{Digital Representation of Images}
  %
  \begin{itemize}
    %
    \item What is digital image? Ultimately, it is individual components that
      are represented as binary numbers!
      %
      \vspace*{-.15in}
      %
    \item The fundamental components of a digital image:
      %
      \begin{itemize}
        %
        \item {\bf Raster (or Bitmap) Image}: individual components are pixels
          %
        \item {\bf Vector Image}: individual components are vector objects
          %
        \item {\bf Trade-offs}: when should you pick one of these formats?
          %
      \end{itemize}
      %
      \vspace*{-.2in}
      %
    \item The pixels of a raster image can be directly manipulated
      %
      \vspace*{-.2in}
      %
    \item Browsers often provide better support for raster images
      %
      \vspace*{-.2in}
      %
    \item But, vector images are resolution independent!
      %
  \end{itemize}
  %
\end{frame}

% Slide
%
\begin{frame}{Models for Representing Colors on the Web}
  %
  \begin{itemize}
    %
    \item We encode a color as a sequence of numbers
      %
      \vspace*{-.2in}
      %
    \item What are the benefits of encoding color as a number?
      %
      \vspace*{-.2in}
      %
    \item Popular color models for web pages:
      %
      \begin{itemize}
        %
        \item {\bf RGB}: Use the hexadecimal {\tt \#RRGGBB} format
          %
        \item {\bf CYMK}: Use cyan, magenta, yellow, and black for print
         %
        \item {\bf HSL}: Use hue, saturation, and lightness in CSS3
         %
        \item We adopt the RGB color model for web development!
         %
      \end{itemize}
      %
      \vspace*{-.25in}
      %
    \item You can also manipulate a color by changing its opacity
      %
      \vspace*{-.25in}
      %
    \item You can display colors differently through use of a gradient
      %
  \end{itemize}
  %
\end{frame}

% Slide
%
\begin{frame}[fragile]
  \frametitle{Example of Numbers that Encode Colors}
  \normalsize
  \begin{minipage}{6in}
    \vspace*{.1in}
    \begin{minted}[mathescape, numbersep=5pt, fontsize=\large]{html}
Hex: #afaf5f
RGB: rgb(175, 175, 95)
HSL: hsl(60, 33.3%, 52.9%)
Most similar:
  darkkhaki
  yellowgreen
  khaki
    \end{minted}
  \end{minipage}
  %
  \vspace*{.1in}
  %
  \begin{center}
    Reference: \url{https://github.com/sharkdp/pastel}
  \end{center}
\end{frame}

% Slide
%
\begin{frame}[fragile]
  \frametitle{Defining Shades in ``Sassy'' CSS Files}
  \normalsize
  \begin{minipage}{6in}
    \vspace*{.1in}
    \begin{minted}[mathescape, numbersep=5pt, fontsize=\normalsize]{css}
$white: #fff !default;
$gray-100: material-color("grey", "100");
$gray-200: material-color("grey", "200");
$gray-300: material-color("grey", "300");
$gray-400: material-color("grey", "400");
$gray-500: material-color("grey", "500");
$gray-600: material-color("grey", "600");
$gray-700: material-color("grey", "700");
$gray-800: material-color("grey", "800");
$gray-900: material-color("grey", "900");
$black: #000 !default;
    \end{minted}
  \end{minipage}
  %
\end{frame}

% Slide
%
\begin{frame}[fragile]
  \frametitle{Defining Colors in ``Sassy'' CSS Files}
  \normalsize
  \begin{minipage}{6in}
    \vspace*{.1in}
    \begin{minted}[mathescape, numbersep=5pt, fontsize=\normalsize]{css}
$blue:   material-color("blue", "800");
$indigo: #6610f2 !default;
$purple: #6f42c1 !default;
$pink:   #e83e8c !default;
$red:    #dc3545 !default;
$orange: material-color("orange", "800");
$yellow: #ffc107 !default;
$green:  #28a745 !default;
$teal:   #20c997 !default;
$cyan:   #17a2b8 !default;
    \end{minted}
  \end{minipage}
  %
\end{frame}

% Slide
%
\begin{frame}[fragile]
  \frametitle{Defining Base Colors for a Web Site in SCSS}
  \normalsize
  \begin{minipage}{6in}
    \vspace*{.1in}
    \begin{minted}[mathescape, numbersep=5pt, fontsize=\large]{css}
$primary:   $orange;
$secondary: $gray-900;
$success:   $green !default;
$info:      $blue;
$warning:   $yellow !default;
$danger:    $red !default;
$light:     $gray-200;
$dark:      darken($secondary, 5%);
    \end{minted}
  \end{minipage}
  %
\end{frame}

% Slide
%
\begin{frame}[fragile]
  \frametitle{Defining Color Gradients in CSS}
  \normalsize
  \begin{minipage}{6in}
    \vspace*{.1in}
    \begin{minted}[mathescape, numbersep=5pt, fontsize=\normalsize]{css}
.lineargradientgreen {
  background-image:
     linear-gradient(green, white);
  border: solid #777;
  border-width: 2pt 2pt 2pt 2pt;
  display: table-cell;
  height: 500px;
  text-align: center;
  vertical-align: middle;
  width: 500px;
}
    \end{minted}
  \end{minipage}
  %
\end{frame}


% Slide
%
\begin{frame}{Basic Concepts in Color Theory}
  %
  \begin{itemize}
    %
    \item Represent colors on a wheel from ``warm'' to ``cool''
      %
      \vspace*{-.2in}
      %
    \item Use the wheel to pick the color scheme for a web site
      %
      \vspace*{-.2in}
      %
    \item Popular color scheme strategies for a web site:
      %
      \begin{itemize}
        %
        \item {\bf Complementary}: pick colors on opposite ends of the color
          wheel
          %
        \item {\bf Analogous}: pick colors next to each other on the color wheel
          %
        \item {\bf Triad}: pick colors in an equilateral triangle on the color
          wheel
          %
        \item {\bf Tetradic}: pick colors in rectangle on the color
          wheel
          %
      \end{itemize}
      %
      \vspace*{-.25in}
      %
    \item Other color schemes are also possible! Use tools!
      %
      \vspace*{-.25in}
      %
    \item What are the accessibility challenges for picking a color scheme? How
      to ensure that a color scheme is accessible?
      %
  \end{itemize}
 %
\end{frame}

% Slide
%
\begin{frame}{Understanding Image Concepts for the Web}
  %
  \begin{itemize}
    %
    \item Color depth is the maximum number of possible colors and image can
      contain. What are the trade-offs in color depth?
      %
      \vspace*{-.2in}
      %
    \item Dithering can create the illusion of more colors in an image
      %
      \vspace*{-.2in}
      %
    \item Possibilities for image color depth:
      %
      \begin{itemize}
        %
        \item {\bf 8-bit or less}: indexed color with no more than 256 colors
          %
        \item {\bf 24-bit}: true color with 16.8 million colors
          %
        \item {\bf 32-bit}: true color with 16.8 million colors and transparency
          %
        \item {\bf 48-bit}: 16-bits per red, green, and blue, but not on web
          %
      \end{itemize}
      %
      \vspace*{-.25in}
      %
    \item Greater color bit-depth will reduce the banding effect
      %
      \vspace*{-.25in}
      %
    \item Resizing an image will often reduce its overall quality!
      %
  \end{itemize}
 %
\end{frame}

% Slide
%
\begin{frame}{Display Resolution and the Number of Pixels}
  %
  \begin{itemize}
    %
    \item Display resolution refers to pixels a device can display
      %
      \vspace*{-.2in}
      %
    \item A resolution must have a height and a width component
      %
      \vspace*{-.2in}
      %
    \item What are the trade-offs associated with image resolution?
      %
      \vspace*{-.2in}
      %
    \item What are the trade-offs associated with device resolution?
      %
      \vspace*{-.2in}
      %
    \item What happens if you change the resolution of an image?
      %
      \begin{itemize}
        %
        \item Increase the resolution of an image
          %
        \item Decrease the resolution of an image
          %
        \item Examples: 1024x768 or 1920x1080
          %
      \end{itemize}
      %
      \vspace*{-.2in}
      %
    \item What is the meaning of 720p or 1080p or 4K? Good grief!
  \end{itemize}
  %
\end{frame}

% Slide
%
\begin{frame}[fragile]
  \frametitle{Using ImageMagick to Inspect Images}
  \normalsize
  \begin{minipage}{6in}
    \vspace*{.1in}
    \begin{minted}[mathescape, numbersep=5pt, fontsize=\large]{text}
  Format: JPEG
  Mime type: image/jpeg
  Class: DirectClass
  Geometry: 640x427+0+0
  Resolution: 72x72
  Print size: 8.88889x5.93056
  Units: PixelsPerInch
  Colorspace: sRGB
  Type: TrueColor
    \end{minted}
  \end{minipage}
  %
\end{frame}

% Slide
%
\begin{frame}[fragile]
  \frametitle{Using ImageMagick to Determine Bit-Depth}
  \normalsize
  \begin{minipage}{6in}
    \vspace*{.1in}
    \begin{minted}[mathescape, numbersep=5pt, fontsize=\large]{text}
  Depth: 8-bit
  Channel depth:
    Red: 8-bit
    Green: 8-bit
    Blue: 8-bit
    \end{minted}
  \end{minipage}
  %
  \vspace*{.15in}
  \begin{center}
    What is the overall bit depth of this image? \\
    What was the resolution of this image? \\
    Was this image stored in a compressed format? \\
    What are the trade-offs of image compression? \\
  \end{center}
  %
\end{frame}

% Slide
%
\begin{frame}{Exploring Image File Formats for the Web}
  %
  \begin{itemize}
    %
    \item There are now many different file formats for images!
      %
      \vspace*{-.2in}
      %
    \item Different file formats used to encode images:
      %
      \begin{itemize}
        %
        \item JPEG: Joint photographic experts group with lossy compression
          %
        \item GIF: Graphic interchange format limited to 8-bits or less
          %
        \item PNG: Portable network graphics format losslessly compressed
          %
        \item SVG: Scalable vector graphics stored in an XML file
          %
      \end{itemize}
      %
      \vspace*{-.2in}
      %
    \item XML stands for the ``eXtensible Markup Language''
      %
      \vspace*{-.2in}
      %
    \item What are the benefits of encoding an image in XML?
      %
      \vspace*{-.2in}
      %
    \item What does it look like to encode an image in an XML file?
      %
      \vspace*{-.2in}
      %
  \end{itemize}
  %
\end{frame}

% Slide
%
\begin{frame}[fragile]
  \frametitle{Using XML to Encode Images in SVG Files}
  \normalsize
  \begin{minipage}{6in}
    \vspace*{-.05in}
    \begin{minted}[mathescape, numbersep=5pt, fontsize=\scriptsize]{text}
 <svg xmlns="http://www.w3.org/2000/svg" viewBox="0 0 512 512"><path
 fill="#ef6c00" d="M459.37 151.716c.325 4.548.325 9.097.325 13.645 0
 138.72-105.583 298.558-298.558 298.558-59.452 0-114.68-17.219-161.137-47.106
 8.447.974 16.568 1.299 25.34 1.299 49.055 0 94.213-16.568
 130.274-44.832-46.132-.975-84.792-31.188-98.112-72.772 6.498.974 12.995 1.624
 19.818 1.624 9.421 0 18.843-1.3
 27.614-3.573-48.081-9.747-84.143-51.98-84.143-102.985v-1.299c13.969 7.797
 30.214 12.67 47.431 13.319-28.264-18.843-46.781-51.005-46.781-87.391 0-19.492
 5.197-37.36 14.294-52.954 51.655 63.675 129.3 105.258 216.365
 109.807-1.624-7.797-2.599-15.918-2.599-24.04 0-57.828 46.782-104.934
 104.934-104.934 30.213 0 57.502 12.67 76.67 33.137 23.715-4.548 46.456-13.32
 66.599-25.34-7.798 24.366-24.366 44.833-46.132 57.827 21.117-2.273 41.584-8.122
 60.426-16.243-14.292 20.791-32.161 39.308-52.628 54.253z"/></svg>
    \end{minted}
  \end{minipage}
  %
  \vspace*{-.1in}
  \begin{center}
      %
    This XML code describes the Twitter icon in the SVG format! \\
    Can you see any color specifications inside of this XML content? \\
    Do you see any tags inside of this XML content? \\
    Tools like Inkscape support the editing of SVG files!
  \end{center}
  %
\end{frame}

% Slide
%
\begin{frame}[fragile]
  \frametitle{Using XML to Encode Images in SVG Files}
  \normalsize
  \begin{minipage}{6in}
    \vspace*{.05in}
    \begin{minted}[mathescape, numbersep=5pt, fontsize=\scriptsize]{text}
<svg xmlns="http://www.w3.org/2000/svg" viewBox="0 0 496 512"><path
fill="#ef6c00" d="M165.9 397.4c0 2-2.3 3.6-5.2 3.6-3.3.3-5.6-1.3-5.6-3.6 0-2
2.3-3.6 5.2-3.6 3-.3 5.6 1.3 5.6 3.6zm-31.1-4.5c-.7 2 1.3 4.3 4.3 4.9 2.6 1 5.6
0 6.2-2s-1.3-4.3-4.3-5.2c-2.6-.7-5.5.3-6.2 2.3zm44.2-1.7c-2.9.7-4.9 2.6-4.6
4.9.3 2 2.9 3.3 5.9 2.6 2.9-.7 4.9-2.6 4.6-4.6-.3-1.9-3-3.2-5.9-2.9zM244.8
8C106.1 8 0 113.3 0 252c0 110.9 69.8 205.8 169.5 239.2 12.8 2.3 17.3-5.6
17.3-12.1 0-6.2-.3-40.4-.3-61.4 0 0-70 15-84.7-29.8 0 0-11.4-29.1-27.8-36.6 0
0-22.9-15.7 1.6-15.4 0 0 24.9 2 38.6 25.8 21.9 38.6 58.6 27.5 72.9 20.9 2.3-16
8.8-27.1 16-33.7-55.9-6.2-112.3-14.3-112.3-110.5 0-27.5 7.6-41.3
23.6-58.9-2.6-6.5-11.1-33.3 2.6-67.9 20.9-6.5 69 27 69 27 20-5.6 41.5-8.5
62.8-8.5s42.8 2.9 62.8 8.5c0 0 48.1-33.6 69-27 13.7 34.7 5.2 61.4 2.6 67.9 16
17.7 25.8 31.5 25.8 58.9 0 96.5-58.9 104.2-114.8 110.5 9.2 7.9 17 22.9 17 46.4 0
33.7-.3 75.4-.3 83.6 0 6.5 4.6 14.4 17.3 12.1C428.2 457.8 496 362.9 496 252 496
113.3 383.5 8 244.8 8zM97.2 352.9c-1.3 1-1 3.3.7 5.2 1.6 1.6 3.9 2.3 5.2 1 1.3-1
1-3.3-.7-5.2-1.6-1.6-3.9-2.3-5.2-1zm-10.8-8.1c-.7 1.3.3 2.9 2.3 3.9 1.6 1 3.6.7
4.3-.7.7-1.3-.3-2.9-2.3-3.9-2-.6-3.6-.3-4.3.7zm32.4 35.6c-1.6 1.3-1 4.3 1.3 6.2
2.3 2.3 5.2 2.6 6.5 1
1.3-1.3.7-4.3-1.3-6.2-2.2-2.3-5.2-2.6-6.5-1zm-11.4-14.7c-1.6 1-1.6 3.6 0 5.9 1.6
2.3 4.3 3.3 5.6 2.3 1.6-1.3 1.6-3.9 0-6.2-1.4-2.3-4-3.3-5.6-2z"/></svg>
    \end{minted}
  \end{minipage}
  %
\end{frame}

% Slide
%
\begin{frame}[fragile]
  \frametitle{MIME Types for Data Types Used on the Web}
  \normalsize
  \begin{minipage}{6in}
    \vspace*{.05in}
    \begin{minted}[mathescape, numbersep=5pt, fontsize=\normalsize]{text}
text/html=firefox.desktop
application/x-extension-htm=firefox.desktop
application/x-extension-html=firefox.desktop
application/x-extension-shtml=firefox.desktop
application/xhtml+xml=firefox.desktop
application/x-extension-xhtml=firefox.desktop
application/x-extension-xht=firefox.desktop
image/jpeg=feh.desktop
image/png=feh.desktop
application/pdf=org.pwmt.zathura.desktop
    \end{minted}
  \end{minipage}
  %
\end{frame}

\end{document}
