\documentclass[14pt,aspectratio=169]{beamer}

\usepackage{pgfpages}
\usepackage{fancyvrb}

\usepackage{tikz}
\usepackage{pgfplots}

\usepackage{minted}
\usemintedstyle{tango}

\usepackage{graphicx}

\usetheme{auriga}
\usecolortheme{auriga}

\setbeamercolor{background canvas}{bg=lightgray}

% define some colors for a consistent theme across slides
\definecolor{red}{RGB}{181, 23, 0}
\definecolor{blue}{RGB}{0, 118, 186}
\definecolor{gray}{RGB}{146, 146, 146}

\title{Web Development: \\ Using Color and Web Media}

\author{{\bf Gregory M. Kapfhammer}}

\institute[shortinst]{{\bf Department of Computer Science, Allegheny College}}

\begin{document}

{
  \setbeamercolor{page number in head/foot}{fg=background canvas.bg}
  \begin{frame}
    \titlepage
  \end{frame}
}

% Slide
%
\begin{frame}{Technical Question}
  %
  \hspace*{.25in}
  %
  \vspace*{.1in}
  %
  \begin{minipage}{4.5in}
    %
    \begin{center}
      %
      {\large How can I use HTML and CSS source code and media creation software to
      effectively add color and media such as images to a web site?}
      %
    \end{center}
    %
  \end{minipage}
  %
  \vspace{2ex}
  %
  \begin{center}
    %
    \small Let's learn how to combine CSS and HTML to create color and images!\\
    \small We will also explore the benefits of using external software to
    create media !\\
    \small Since web development in cumulative, please review all previous content!\\
    %
  \end{center}
  %
\end{frame}

% Slide
%
\begin{frame}{Digital Representation of Images}
  %
  \begin{itemize}
    %
    \item What is digital image? Ultimately, it is individual components that
      are represented as binary numbers!
      %
      \vspace*{-.15in}
      %
    \item The fundamental components of a digital image:
      %
      \begin{itemize}
        %
        \item {\bf Raster (or Bitmap) Image}: individual components are pixels
          %
        \item {\bf Vector Image}: individual components are vector objects
          %
        \item {\bf Trade-offs}: when should you pick one of these formats?
          %
      \end{itemize}
      %
      \vspace*{-.2in}
      %
    \item The pixels of a raster image can be directly manipulated
      %
      \vspace*{-.2in}
      %
    \item Browsers often provide better support for raster images
      %
      \vspace*{-.2in}
      %
    \item But, vector images are resolution independent!
      %
  \end{itemize}
  %
\end{frame}

% Slide
%
\begin{frame}{Models for Representing Colors on the Web}
  %
  \begin{itemize}
    %
    \item We encode a color as a sequence of numbers
      %
      \vspace*{-.2in}
      %
    \item What are the benefits of encoding color as a number?
      %
      \vspace*{-.2in}
      %
    \item Popular color models for web pages:
      %
      \begin{itemize}
        %
        \item {\bf RGB}: Use the hexadecimal {\tt \#RRGGBB} format
          %
        \item {\bf CYMK}: Use cyan, magenta, yellow, and black for print
         %
        \item {\bf HSL}: Use hue, saturation, and lightness in CSS3
         %
        \item We adopt the RGB color model for web development!
         %
      \end{itemize}
      %
      \vspace*{-.25in}
      %
    \item You can also manipulate a color by changing its opacity
      %
      \vspace*{-.25in}
      %
    \item You can display colors differently through use of a gradient
      %
  \end{itemize}
  %
\end{frame}

% Slide
%
\begin{frame}[fragile]
  \frametitle{Defining Shades in ``Sassy'' CSS Files}
  \normalsize
  \begin{minipage}{6in}
    \vspace*{.1in}
    \begin{minted}[mathescape, numbersep=5pt, fontsize=\normalsize]{css}
$white: #fff !default;
$gray-100: material-color("grey", "100");
$gray-200: material-color("grey", "200");
$gray-300: material-color("grey", "300");
$gray-400: material-color("grey", "400");
$gray-500: material-color("grey", "500");
$gray-600: material-color("grey", "600");
$gray-700: material-color("grey", "700");
$gray-800: material-color("grey", "800");
$gray-900: material-color("grey", "900");
$black: #000 !default;
    \end{minted}
  \end{minipage}
  %
\end{frame}

% Slide
%
\begin{frame}[fragile]
  \frametitle{Defining Colors in ``Sassy'' CSS Files}
  \normalsize
  \begin{minipage}{6in}
    \vspace*{.1in}
    \begin{minted}[mathescape, numbersep=5pt, fontsize=\normalsize]{css}
$blue:   material-color("blue", "800");
$indigo: #6610f2 !default;
$purple: #6f42c1 !default;
$pink:   #e83e8c !default;
$red:    #dc3545 !default;
$orange: material-color("orange", "800");
$yellow: #ffc107 !default;
$green:  #28a745 !default;
$teal:   #20c997 !default;
$cyan:   #17a2b8 !default;
    \end{minted}
  \end{minipage}
  %
\end{frame}

% Slide
%
\begin{frame}[fragile]
  \frametitle{Defining Base Colors for a Web Site}
  \normalsize
  \begin{minipage}{6in}
    \vspace*{.1in}
    \begin{minted}[mathescape, numbersep=5pt, fontsize=\large]{css}
$primary:   $orange;
$secondary: $gray-900;
$success:   $green !default;
$info:      $blue;
$warning:   $yellow !default;
$danger:    $red !default;
$light:     $gray-200;
$dark:      darken($secondary, 5%);
    \end{minted}
  \end{minipage}
  %
\end{frame}

% Slide
%
\begin{frame}[fragile]
  \frametitle{Defining Color Gradients in CSS}
  \normalsize
  \begin{minipage}{6in}
    \vspace*{.1in}
    \begin{minted}[mathescape, numbersep=5pt, fontsize=\normalsize]{css}
.lineargradientgreen {
  background-image:
     linear-gradient(green, white);
  border: solid #777;
  border-width: 2pt 2pt 2pt 2pt;
  display: table-cell;
  height: 500px;
  text-align: center;
  vertical-align: middle;
  width: 500px;
}
    \end{minted}
  \end{minipage}
  %
\end{frame}


\end{document}
