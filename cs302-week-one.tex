\documentclass[14pt,aspectratio=169]{beamer}

\usepackage{pgfpages}
\usepackage{fancyvrb}
\usepackage{tikz}
\usepackage{pgfplots}

\usetheme{auriga}
\usecolortheme{auriga}

\setbeamercolor{background canvas}{bg=lightgray}

% define some colors for a consistent theme across slides
\definecolor{red}{RGB}{181, 23, 0}
\definecolor{blue}{RGB}{0, 118, 186}
\definecolor{gray}{RGB}{146, 146, 146}

\title{Web Development: \\ Web Programming Tools}

\author{{\bf Gregory M. Kapfhammer}}

\institute[shortinst]{{\bf Department of Computer Science, Allegheny College}}

\begin{document}

{
  \setbeamercolor{page number in head/foot}{fg=background canvas.bg}
  \begin{frame}
    \titlepage
  \end{frame}
}

%% Slide
%
\begin{frame}{Technical Question}
  %
  \hspace*{.25in}
  \begin{minipage}{4.5in}
    %
    \begin{center}
      %
      {\large How do I install and use the industry-standard programming tools
        that will help me to design, implement, test, and maintain mobile-ready web
      sites?}
      %
    \end{center}
    %
  \end{minipage}
  %
  \vspace{3ex}
  %
  \begin{center}
    %
    \small Let's learn more about version control, text editors, Docker, the
    Markdown language for technical writing, and web programming languages!
    %
  \end{center}
  %
\end{frame}

% Slide
%
\begin{frame}{Version Control with Git and GitHub}
%
  \begin{itemize}
    %
    \item Benefits of version control systems
      %
    \item Using Git and GitHub
      \begin{itemize}
        \item Git tracks versions of files in directories
        \item GitHub enables sharing and collaboration
        \item Industry standard tools used in all assignments
      \end{itemize}
      %
    \item Challenges of version control systems
      %
  \end{itemize}
%
\end{frame}

% Slide
%
\begin{frame}{Editing Web Languages with VSCode}
%
  \begin{itemize}
    %
    \item Popularity of VSCode according to StackOverflow
      %
    \item Using the VSCode development environment
      \begin{itemize}
        \item Edit Markdown, HTML, CSS, and JavaScript files
        \item Commit changes to a GitHub repository
        \item Remote collaboration with the Live Share extension
      \end{itemize}
      %
    \item Get started by downloading VSCode from \url{https://code.visualstudio.com/}
      %
  \end{itemize}
%
\end{frame}

% Slide
%
\begin{frame}{Running GatorGrader and Tools with Docker}
%
  \begin{itemize}
    %
    \item Popularity of Docker according to StackOverflow
      %
    \item Using the Docker during web development
      \begin{itemize}
        \item Download a Docker container from DockerHub
        \item Use a Docker container once to grade the work repository
        \item Enter into a Docker container to perform multiple commands
      \end{itemize}
      %
    \item Get started by downloading Docker from \url{https://docs.docker.com/desktop/}
      %
  \end{itemize}
%
\end{frame}

% Slide
%
\begin{frame}{Using Markdown and GitHub for Documentation}
%
  \begin{itemize}
    %
    \item A language for describing how to format rendered text
      %
    \item Benefits of using Markdown for documentation
      \begin{itemize}
        \item Simple and powerful approach to technical writing
        \item Formats both standard textual elements and source code
        \item GitHub natively renders Markdown in a Git repository
      \end{itemize}
      %
    \item Learn more about how to use Markdown by reading \url{https://www.markdownguide.org/}
      %
  \end{itemize}
%
\end{frame}

% Slide
%
\begin{frame}{Using HTML, CSS, and JavaScript}
%
  \begin{itemize}
    %
    \item Multiple languages in the web development ``stack''
      %
    \item Role that each language plays in a web site
      \begin{itemize}
        \item HTML: Content and tags that assign meaning to the content
        \item CSS: Rules and queries that style the HTML content
        \item JavaScript: Functions that facilitate user interaction
      \end{itemize}
      %
    \item Code at \url{https://github.com/gkapfham/www.gregorykapfhammer.com/}
      %
  \end{itemize}
%
\end{frame}

%% Slide
%
\begin{frame}{Technical Question}
  %
  \hspace*{.25in}
  \begin{minipage}{4.5in}
    %
    \begin{center}
      %
      {\large How do I install and use the industry-standard programming tools
        that will help me to design, implement, test, and maintain mobile-ready web
      sites?}
      %
    \end{center}
    %
  \end{minipage}
  %
  \vspace{3ex}
  %
  \begin{center}
    %
    \small Let's learn some useful Git commands and then consider the next steps!
    %
  \end{center}
  %
\end{frame}

% Slide
%
\begin{frame}{Creating a Clone of a GitHub Repository}
  %
  \setlength{\leftmarginii}{0.5cm}
  %
  \begin{itemize}
    %
    \item ``{\tt git clone}'' transfers from GitHub to your computer
      %
    \item Cloning a Git repository on GitHub
      %
      {\tiny
        \begin{itemize}
          \item {\tt git clone git@github.com:Allegheny-Computer-Science-302-F2020/cs302-F2020-plans.git}
            %
          \item {\tt git clone https://github.com/Allegheny-Computer-Science-302-F2020/cs302-F2020-plans.git}
        \end{itemize}
      }
      %
    \item What are the differences in these ``{\tt git clone}'' commands? What are the trade-offs in using them?
      %
  \end{itemize}
  %
\end{frame}

% Slide
%
\begin{frame}{Handy Commands to Support GitHub Use}
  %
  \begin{itemize}
    %
    \item ``{\tt git push -u origin master}'' transfers content from your computer
      to the GitHub servers
      %
    \item ``{\tt git commit cs302-week-one.tex}'' commits the changes in the
      file to your GitHub repository
      %
    \item ``{\tt git checkout -b feat/add-week-one-slides}'' creates a branch in
      a GitHub repository
      %
  \end{itemize}
  %
\end{frame}

% Slide
%
\begin{frame}{Mastering These Software Tools}
  %
  \begin{itemize}
    %
    \item Install VS Code, participate in a live share, edit Markdown
      %
    \item Install Git command-line client, use in terminal \& VS Code
      %
    \item Clone a GitHub repository and open its code in VS Code
      %
    \item Run a web server and track its behavior in your terminal
      %
  \end{itemize}
  %
\end{frame}

\end{document}
