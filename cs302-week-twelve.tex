\documentclass[14pt,aspectratio=169]{beamer}

\usepackage{pgfpages}
\usepackage{fancyvrb}

\usepackage{tikz}
\usepackage{pgfplots}

\usepackage{minted}
\usemintedstyle{tango}

\usepackage{graphicx}

\usetheme{auriga}
\usecolortheme{auriga}

\setbeamercolor{background canvas}{bg=lightgray}

% define some colors for a consistent theme across slides

\definecolor{red}{RGB}{181, 23, 0}
\definecolor{blue}{RGB}{0, 118, 186}
\definecolor{gray}{RGB}{146, 146, 146}

\title{Web Development: \\ Understanding and Applying\\ Security Principles}

\author{{\bf Gregory M. Kapfhammer}}

\institute[shortinst]{{\bf Department of Computer Science, Allegheny College}}

\begin{document}

{
  \setbeamercolor{page number in head/foot}{fg=background canvas.bg}
  \begin{frame}
    \titlepage
  \end{frame}
}

% Slide
%
\begin{frame}{Technical Question}
  %
  \hspace*{.15in}
  %
  \begin{minipage}{5in}
    %
    \vspace*{.25in}
    %
    \begin{center}
      %
      {\large How can I apply security principles and follow security best
      practices to design and implement web sites that are secure?}
      %
    \end{center}
    %
  \end{minipage}
  %
  \vspace{2ex}
  %
  \begin{center}
    %
    \small This week's focus is less on teaching you how to write secure code
    and more about raising your awareness concerning the principles and best
    practices in the field of computer security. Try to find ways to apply these
    principles to the development of your own web sites! \\
    %
  \end{center}
  %
\end{frame}

% Slide
%
\begin{frame}{You Need to ``Build in Security'' for the Web}
  %
  \begin{itemize}
    %
    \item It is tempting to ``bolt on'' security after finishing a web site
      %
      \vspace*{-.15in}
      %
    \item We can make security mistakes throughout development!
      %
      \vspace*{-.15in}
      %
    \item Patched systems are often less elegant and maintainable
      %
      \vspace*{-.15in}
      %
    \item Web sites also have many attack vectors to consider
      %
      \vspace*{-.15in}
      %
    \item {\bf Information Security}: practice of protecting information
      %
      \vspace*{-.15in}
      %
    \item {\bf Information Assurance}: ensures that data is not lost
      %
      \vspace*{-.15in}
      %
    \item What are the best practices that promote secure web sites? What are
      the tools and protocols that aid security?
      %
  \end{itemize}
  %
\end{frame}

% Slide
%
\begin{frame}{Pros and Cons of Software Frameworks}
  %
  \begin{itemize}
    %
    \item {\bf ISO/IEC Security Standard}: a detailed security standard
      containing details about risk assessment and management, security
      policies, and security best practices
      %
      \vspace*{-.15in}
      %
    \item The {\bf CIA} triad of information security
      %
      \begin{itemize}
        %
        \item {\bf Confidentiality}: maintain privacy of all site data
          %
        \item {\bf Integrity}: ensure data is accurate and correct
          %
        \item {\bf Availability}: make data available to authorized people
          %
      \end{itemize}
      %
      \vspace*{-.25in}
      %
    \item {\bf High Availability Web Site}: A web site that contains redundant
      systems that can ensure a high uptime
      %
      \vspace*{-.25in}
      %
    \item Changelog podcast hosts take down their own web site to show
      availability: \url{https://bit.ly/3pFDe1E}
      %
  \end{itemize}
  %
\end{frame}

% Slide
%
\begin{frame}{Security Threats Faced by Web Sites}
  %
  \begin{itemize}
    %
    \item {\bf Security Threat}: a particular path that a hacker could use to
      exploit a vulnerability and gain unauthorized access
      %
      \vspace*{-.15in}
      %
    \item The {\bf STRIDE} threat classification system:
      %
      \begin{itemize}
        %
        \item {\bf Spoofing}: use another's information to access site
          %
        \item {\bf Tampering}: modify site data in authorized fashion
          %
        \item {\bf Repudiation}: remove all traces of attack on site
          %
        \item {\bf Information disclosure}: access non-permitted data
          %
        \item {\bf Denial of Service}: prevent real users from attacking system
          %
        \item {\bf Elevation of Privilege}: increase rights to access site
          %
      \end{itemize}
      %
      \vspace*{-.25in}
      %
    \item {\bf Security Vulnerabilities}: the security ``holes'' in your system
      %
  \end{itemize}
  %
\end{frame}

% Slide
%
\begin{frame}{Authentication and Authorization}
  %
  \begin{itemize}
    %
    \item {\bf Authentication}: is a user of the site who they say they are?
      %
      \vspace*{-.4in}
      %
    \item {\bf Authorization}: can a user of site perform an action?
      %
      \vspace*{-.15in}
      %
    \item {\bf Security Policies} for a web site:
      %
      \begin{itemize}
        %
        \item {\bf Usage Policy}: what systems a person can use
          %
        \item {\bf Authentication Policy}: how users are granted access
          %
        \item {\bf Legal Policy}: binding rules for maintainers and users
          %
      \end{itemize}
      %
      \vspace*{-.2in}
      %
    \item Are password rotation policies effective security policies?
      %
      \vspace*{-.2in}
      %
    \item Should every person be required to use two-factor auth?
      %
      \vspace*{-.2in}
      %
    \item What are the cost-benefit trade-offs of different policies?
      %
  \end{itemize}
  %
\end{frame}

% Slide
%
\begin{frame}{Security by Design for Web Sites}
  %
  \begin{itemize}
    %
    \item {\bf Secure by Design}: apply software engineering principles
      throughout the life cycle of web site development
      %
      \vspace*{-.15in}
      %
    \item {\bf Requirements}: privacy needs, security policy, CIA triad
      %
      \vspace*{-.15in}
      %
    \item {\bf Design}: threat and risk assessment, redundancy planning
      %
      \vspace*{-.15in}
      %
    \item {\bf Implementation}: pairs, code reviews, defensive code
      %
      \vspace*{-.15in}
      %
    \item {\bf Testing}: security and vulnerability tests, correctness
      %
      \vspace*{-.15in}
      %
    \item {\bf Deployment}: penetration testing, internal attacks
      %
      \vspace*{-.15in}
      %
    \item Take the software engineering course to learn more!
      %
  \end{itemize}
  %
\end{frame}

% Slide
%
\begin{frame}{User Authentication for the Web}
  %
  \begin{itemize}
    %
    \item {\bf Single-factor} versus {\bf two-factor} authentication
      %
      \vspace*{-.15in}
      %
    \item {\bf Authentication Factors} for a web site:
      %
      \begin{itemize}
        %
        \item {\bf Knowledge}: information belonging to a specific person
          %
        \item {\bf Ownership}: items that a specific person possesses
          %
        \item {\bf Inherence}: intrinsic properties connected to a
          person
          %
      \end{itemize}
      %
      \vspace*{-.25in}
      %
    \item {\bf Browsers} and {\bf forms} can provide authentication!
      %
      \vspace*{-.25in}
      %
    \item {\bf Oauth} protocol provides a standard way to authorize a user for a
      web site or a mobile application
      %
      \vspace*{-.25in}
      %
    \item User authentication is challenging! Rely on previously vetted tools!
      Use different factors whenever possible!
      %
  \end{itemize}
  %
\end{frame}

% Slide
%
\begin{frame}{User Authorization for the Web}
  %
  \begin{itemize}
    %
    \item {\bf Authorization} defines the rights, privileges, and actions a
      person has when interacting with a web site
      %
      \vspace*{-.15in}
      %
    \item Authentication gives access, authorization allows behavior
      %
      \vspace*{-.15in}
      %
    \item {\bf Principle of Least Privilege} for a web site's server:
      %
      \begin{itemize}
        %
        \item Define different privilege levels for types of users
          %
        \item Start a user at the ``least'' level of privilege
          %
        \item Elevate privilege only when required for an action
          %
      \end{itemize}
      %
      \vspace*{-.25in}
      %
    \item The {\bf root} and {\bf sudo} commands on a web server
      %
      \vspace*{-.25in}
      %
    \item The benefits and drawbacks of running a process as root?
      %
      \vspace*{-.25in}
      %
    \item {\bf Security audit} to assess authentication and authorization
      %
  \end{itemize}
  %
\end{frame}

% Slide
%
\begin{frame}{Common Threat Vectors on the Web}
  %
  \begin{itemize}
    %
    \item There are many ways in which a person can attack a site!
      %
      \vspace*{-.15in}
      %
    \item Common threat vectors for a web site:
      %
      \begin{itemize}
        %
        \item {\bf Brute Force}: keep trying until you guess correctly
          %
        \item {\bf SQL Injection}: insert revealing SQL into a form
          %
        \item {\bf Cross-site Scripting}: put malicious code in trustworthy
          site
          %
        \item {\bf Insecure Data}: make secure data publicly available
          %
        \item {\bf Denial of Service}: overload a server with requests
          %
        \item {\bf Security Misconfiguration}: run out-of-date software
          %
        \item {\bf Arbitrary Program Execution}: run unchecked commands
          %
      \end{itemize}
      %
      \vspace*{-.25in}
      %
    \item What can you do to ensure that your sites are secure?
      %
  \end{itemize}
  %
\end{frame}

\end{document}
